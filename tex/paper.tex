
% Version history:





\title{A Paper framework}
%
\author{Carsten~Wulff, \textit{Member, IEEE} }

\maketitle

\begin{abstract}
This is a framework for writing papers, I've used it for JSSC publications. Feel free to use it for what you want.

Carsten Wulff (Corresponding author), carste@wulff.no

\end{abstract}
\begin{IEEEkeywords}
latex
\end{IEEEkeywords}

\section{Introduction} \label{introduction}


\IEEEPARstart{W}{riting} beautiful papers is can be split into two, typesetting and content. I can't help you on the content, but this framework is something that's been developed over the years to do the typesetting with latex.

\section{Getting Started}

\begin{lstlisting}[frame=single,style=paperBashStyle]
mkdir <whatever your paper is called>
cd <whatever your paper is called>
git clone https://github.com/wulffern/paper
make -f paper/Makfile/Makefile install
make all
\end{lstlisting}

\section{Directories}

The "tex" directory contain the source files. It also will contain a couple examples on how to compile figures in tikz and circuitikz.


\section{Figures}

I would recommend that you make your figures using tikz. Yes, it takes time, but it's the only
way I've found to make beautiful pictures. Like the methodology used in \cite{Wulff17} where I published \cite{ciccreator16}

\begin{figure}[tb]
\centerline{\includegraphics[width=\myfigwidth]{fig_process}}
\caption{\secedit{Design methodology}}
\label{fig_cic}
\end{figure}

Or the circuit tikz pictures that can be made, which was used in the same paper
\begin{figure}[tb]
\centerline{\includegraphics[width=\myfigwidth]{fig_comparator}}
\caption{\edit{Strong-arm comparator with kick-back
  compensation. Transistors without numbers are unit size, while
  transistors with numbers are parallel combinations of unit transistors.}}
\label{fig_comp}
\end{figure}
